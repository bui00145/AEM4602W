%\title{Overleaf Memo Template}
% Using the texMemo package by Rob Oakes
\documentclass[a4paper,11pt]{texMemo}
\usepackage[english]{babel}
\usepackage{graphicx, lipsum}

%% Edit the header section here. To include your
%% own logo, upload a file via the files menu.
%\memoto{}
%If you want you can uncomment the line above to address the memo to one of the instructors
\memofrom{Richard Bui}
\memosubject{Lift and Drag Curve Based on the Angle of Attack Prelab Memo}
\memodate{\today}
\logo{\includegraphics[width=0.3\textwidth]{UMN_lgM-D2D-wdmk-maroon-blk.png}}

\begin{document}
\maketitle

%% Main Memo
The purpose of the lab is to characterize the $C_L$ and $C_D$ curves along different angles of attack $(\alpha)$ of different NACA airfoils. The lab will be conducted in Akerman's wind tunnel, characterizing the $C_L$ and $C_D$ curves using smaller mock NACA airfoils. The purpose of this lab is to show that the characterization of airfoils is possible with smaller scales of the same airfoils. To find $C_L$ and $C_D$, two forces acting on the airfoil will be measured, the normal force and the axial force of the airfoil. The two force measurements are used to calculate the lift and drag of the airfoil, which will eventually be used to calculate $C_L$ and $C_D$. The measurements of the two forces will be done from a strain gauge from the sting. The strain gauge will measure the normal and axial force of the airfoil as voltage. The voltage of the strain gauge will be measured using an oscilloscope, with the voltage recorded, and then processed to find the force of the normal and axial forces. 

Since the angle of attack will be changed to measure the forces of the airfoil, there are some lab procedures that can be done to maximize time. Since the $C_L$ vs. $\alpha$ curve is mostly linear up to the angle of attack stall ($\alpha_{stall}$), fewer points will be taken up to near $\alpha_{stall}$, since the trend is linear. More points will be taken as the $\alpha_{stall}$ approach, as one of the main objectives of this lab is to see the effects of lift and drag as the stall approach. It is also known that as the angle of attack approaches stall, the $C_L$ is not linear. 

\end{document}